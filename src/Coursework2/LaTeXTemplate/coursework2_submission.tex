\documentclass[a4,10pt,fleqn]{article}  %@@
\usepackage{graphicx}
\usepackage{amsmath,amssymb}
\oddsidemargin -0.2in %-0.4in for smaller margins
\topmargin -1.0in %-1in (for above)
\textheight 9.8in %10.75in (for above)
\textwidth 6.75in %6.75in (for above)
\parindent=0pt
\usepackage{caption}

%\pagestyle{headings}%{headings}if you want page numbers top right

%***************************************************************************
%\pagestyle{myheadings}for header and pagenumber with \markright{\rm...{  }}
%e.g.  \markright{\rm{HG123\ Fourier}}
%use with \thispagestyle{empty} after begin document

%***************************************************************************
\newcommand{\ds}{\displaystyle}%display maths
\newcommand{\sst}{\scriptstyle}%very small maths
\newcommand{\sm}{\textstyle}%normal size maths
\newcommand{\bm}[1]{\mbox{\boldmath{$#1$}}}
\newcommand{\rom}[1]{\mbox{${\rm#1}$}}%roman in maths
\newcommand{\pa}{\partial}
\newcommand{\ep}{\varepsilon}
\newcommand{\na}{\bm\nabla}
\newcommand{\blR}{\rom{I\!R}}
\newcommand{\blI}{\rom{I\!I}}
\newcommand{\there}{$\dot{.\ .}$}
\renewcommand{\it}{\textit}
\renewcommand{\sc}{\textsc}
\renewcommand{\rm}{\textrm}
\newcommand{\ve}[1]{\mbox{$#1$}}
\newcommand{\veT}[1]{\mbox{$#1$}^{\operatorname{T}}}
\newcommand{\tr}{\mbox{Tr}}
\newcommand{\R}{\mathbb{R}}
\renewcommand{\d}{\mathrm{d}}
\newcommand{\norm}[1]{|\!| #1 |\!|}
\newcommand{\ip}[2]{<#1,#2>}
\newcommand{\dd}{\mbox{\,d}}


\def\be{\begin{equation}}
\def\ee{\end{equation}}
\def\ba{\begin{array}}
\def\ea{\begin{array}}
\newcommand{\bea}{\begin{eqnarray}}
\newcommand{\eea}{\end{eqnarray}}
\newcommand{\beas}{\begin{eqnarray*}}
\newcommand{\eeas}{\end{eqnarray*}}
\newcommand{\beg}{\begin{eqngroup}}
\newcommand{\eeg}{\end{eqngroup}}
\newcommand{\nno}{\nonumber}
\def \ptl{\partial}

\newcommand{\vect}[1]{\bm{#1}}
\newcommand{\vu}{\vect{u}}
\newcommand{\vf}{\vect{f}}
\newcommand{\verr}{\vect{e}}
\newcommand{\vk}{\vect{k}}
\newcommand{\il}[1]{\texttt{#1}}

\renewcommand{\familydefault}{\sfdefault}

%***************************************************************************
\begin{document}
% \sffamily%@@
\thispagestyle{empty}
\begin{center}
University of Nottingham\\
School of Mathematical Sciences
\end{center}
%***************************************************************************
\large\textsc {MATH4063/G14SCC}\hfill\large\textsc {Scientific Computing and C++}%dont forget to
%change these
\vspace*{2ex}
\hrule
\vskip0.25cm
\textbf{Deadline: 8th January 2021, 3:00pm (GMT)}
\hfill \textbf{Coursework 2 -- Solution Template}\\

\noindent 
{\em Your solutions to the assessed coursework may be submitted using this template.
Please cut and paste the output from your codes into the correct parts of this file
and include your plots and responses to the questions where suggested. Once this
template has been completed, you must then create a pdf file for submission. Under
Windows or Mac you can use Texmaker + a LaTeX compiler; from the Windows Virtual
Desktop this may be accessed as follows:
\begin{verbatim}
Start > UoN Application > (UoN) Texmaker 5
\end{verbatim}
Open this file under {\tt File}; to build the pdf file, click the arrow next to
{\tt Quick Build}; this will then generate the file {\tt coursework2\_submission.pdf}. 

\vspace{2mm}
 You may use an alternative document processing system, such as Word, to produce
a pdf file containing your results, plots and answers. However, if you do, you must
format your answers in the same way as suggested below.

\vspace{2mm}
 A single zip or tar file containing the file {\tt coursework2\_submission.pdf}
and all the files in the requested folders in the checklists below should be
submitted on Moodle. Note that all parameters and values should be set within
your codes: do NOT use inputs such as those obtained with {\tt std::cin} or from
the command line.
}

\vspace{25mm}
\centerline{\LARGE STUDENT NAME}

\clearpage

\section*{Question 1(c)}

\textbf{File checklist for folder \il{Q1}:}
\begin{itemize}
 \item \il{AbstractApproximator.cpp}, \il{AbstractApproximator.hpp}
 \item \il{Driver.cpp}
 \item \il{Lagrange.cpp}, \il{Lagrange.hpp}
 \item \il{Vector.cpp}, \il{Vector.hpp}
 \item For any additional files, provide a \il{README.txt}
\end{itemize}

\vspace{\baselineskip}
\begin{enumerate}

\item[1(c)] Enter your output here (max 1 page, display only selected output if necessary).

\begin{verbatim}
%%%%%%%%%%%%%%%%%%% Output of Driver.cpp %%%%%%%%%%%%%%%%%%%%%%%%%%%%%%%%

%%%%%%%%%%%%%%%%%%%%%%%%%%%%%%%%%%%%%%%%%%%%%%%%%%%%%%%%%%%%%%%%%%%%%%%%%
\end{verbatim}

\item[1(c)] Include your plots and comments here.

%\begin{center}
% \hbox{\hspace*{-10mm} \includegraphics[width=0.6\textwidth]{f1_comparison} \hspace*{-15mm}
%       \includegraphics[width=0.6\textwidth]{f2_comparison}}
% \hbox{\hspace*{-10mm} \includegraphics[width=0.6\textwidth]{f3_comparison} \hspace*{-15mm}
%       \includegraphics[width=0.6\textwidth]{f4_comparison}}
%\end{center}

\end{enumerate}

\clearpage

\section*{Question 2(c)}

\textbf{File checklist for folder \il{Q2}:}
\begin{itemize}
 \item \il{AbstractQuadratureRule.cpp}, \il{AbstractQuadratureRule.hpp}
 \item \il{Driver.cpp}
 \item \il{Gauss4point.cpp}, \il{Gauss4point.hpp}
 \item \il{Matrix.cpp}, \il{Matrix.hpp}
 \item \il{Simpson.cpp}, \il{Simpson.hpp}
 \item \il{Vector.cpp}, \il{Vector.hpp}
 \item For any additional files, provide a \il{README.txt}
\end{itemize}

\vspace{\baselineskip}
\begin{enumerate}

\item[2(c)] Enter your output here (display only selected output if necessary).

\begin{verbatim}
%%%%%%%%%%%%%%%%%%% Output of Driver.cpp %%%%%%%%%%%%%%%%%%%%%%%%%%%%%%%%

%%%%%%%%%%%%%%%%%%%%%%%%%%%%%%%%%%%%%%%%%%%%%%%%%%%%%%%%%%%%%%%%%%%%%%%%%
\end{verbatim}

\item[2(c)] Include your plots and comments here.

\end{enumerate}

\clearpage

\section*{Question 3(b)}

\textbf{File checklist for folder \il{Q3}:}
\begin{itemize}
 \item \il{AbstractApproximator.cpp}, \il{AbstractApproximator.hpp}
 \item \il{AbstractQuadratureRule.cpp}, \il{AbstractQuadratureRule.hpp}
 \item \il{BestL2Fit.cpp}, \il{BestL2Fit.hpp}
 \item \il{Driver.cpp}
 \item \il{Gauss4point.cpp}, \il{Gauss4point.hpp}
 \item \il{Matrix.cpp}, \il{Matrix.hpp}
 \item \il{Vector.cpp}, \il{Vector.hpp}
 \item For any additional files, provide a \il{README.txt}
\end{itemize}

\vspace{\baselineskip}
\begin{enumerate}

\item[3(b)] Enter your output here (display only selected output if necessary).

\begin{verbatim}
%%%%%%%%%%%%%%%%%%% Output of Driver.cpp %%%%%%%%%%%%%%%%%%%%%%%%%%%%%%%%

%%%%%%%%%%%%%%%%%%%%%%%%%%%%%%%%%%%%%%%%%%%%%%%%%%%%%%%%%%%%%%%%%%%%%%%%%
\end{verbatim}

\item[3(b)] Include your plots and comments here.

%\begin{center}
% \hbox{\hspace*{-10mm} \includegraphics[width=0.6\textwidth]{f1_comparison} \hspace*{-15mm}
%       \includegraphics[width=0.6\textwidth]{f2_comparison}}
% \hbox{\hspace*{-10mm} \includegraphics[width=0.6\textwidth]{f3_comparison} \hspace*{-15mm}
%       \includegraphics[width=0.6\textwidth]{f4_comparison}}
%\end{center}

\end{enumerate}

\clearpage

\section*{Question 4(b)}

\textbf{File checklist for folder \il{Q4}:}
\begin{itemize}
 \item \il{AbstractApproximator.cpp}, \il{AbstractApproximator.hpp}
 \item \il{AbstractQuadratureRule.cpp}, \il{AbstractQuadratureRule.hpp}
 \item \il{Driver.cpp}
 \item \il{Gauss4point.cpp}, \il{Gauss4point.hpp}
 \item \il{LocalBestL2Fit.cpp}, \il{LocalBestL2Fit.hpp}
 \item \il{Matrix.cpp}, \il{Matrix.hpp}
 \item \il{Vector.cpp}, \il{Vector.hpp}
 \item For any additional files, provide a \il{README.txt}
\end{itemize}

\vspace{\baselineskip}
\begin{enumerate}

\item[4(b)] Enter your output here (display only selected output if necessary).

\begin{verbatim}
%%%%%%%%%%%%%%%%%%% Output of Driver.cpp %%%%%%%%%%%%%%%%%%%%%%%%%%%%%%%%

%%%%%%%%%%%%%%%%%%%%%%%%%%%%%%%%%%%%%%%%%%%%%%%%%%%%%%%%%%%%%%%%%%%%%%%%%
\end{verbatim}

\item[4(b)] Include your plots and comments here.

%\begin{center}
% \hbox{\hspace*{-10mm} \includegraphics[width=0.6\textwidth]{f1_comparison} \hspace*{-15mm}
%       \includegraphics[width=0.6\textwidth]{f2_comparison}}
% \hbox{\hspace*{-10mm} \includegraphics[width=0.6\textwidth]{f3_comparison} \hspace*{-15mm}
%       \includegraphics[width=0.6\textwidth]{f4_comparison}}
%\end{center}

\end{enumerate}

\end{document}
